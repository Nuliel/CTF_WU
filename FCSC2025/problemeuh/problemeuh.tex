\documentclass[]{article}

\usepackage{amsmath}
\usepackage{minted}
\usepackage{xcolor}

\setlength\parindent{0pt}

\definecolor{LightGray}{gray}{0.9}
\setminted[python]{
frame=lines,
framesep=2mm,
baselinestretch=1.2,
bgcolor=LightGray,
fontsize=\footnotesize,
linenos
}

\title{Write-up Problèmeuh}
\author{Nuliel}
\date{}

\begin{document}
\maketitle
Problèmeuh is a crypto challenge from FCSC 2025. The goal is to solve a system of equations, with both linear and quadratic equations.

\section{Problem statement}

\begin{quote}
    Here is a nice and small system to solve.
\end{quote}

And the python script attached:

\inputminted{python}{problemeuh.py}

\section{Solution}

We have this system of equations:

\begin{equation*}
    \begin{cases}
        a = 487 c \\
        159 a = 485 b \\
        x^2 = a + b \\
        y (3 y - 1) = 2 b
    \end{cases}
\end{equation*}

\subsection{Two first equations}

We multiply the first equation by $159$:
\begin{equation*}
    \begin{cases}
        159 a = 159 \cdot 487 c \\
        159 a = 485 b
    \end{cases}
\end{equation*}
So we have 
\begin{align*}
    159 \cdot 487 c = 485 b
\end{align*}

As $159$, $485$ and $487$ are coprime, we must have
\begin{itemize}
    \item 159 and 487 in the factors of b
    \item 485 in the factors of c
\end{itemize}
From this fact, we can express $a$, $b$ and $c$ in function of only one unknown $k$:

\begin{equation*}
    \begin{cases}
        b = 159 \cdot 487 k \\
        c = 485 k \\
        a = 487 \cdot 485 k
    \end{cases}
\end{equation*}

\subsection{Third equation}

We can replace, develop and factor in this equation:

\begin{align*}
    x^2 &= a + b \\
        &= k \cdot (487 \cdot 485 + 159 \cdot 487) \\
        &= k \cdot (2^2 \cdot 7 \cdot 23 \cdot 487)
\end{align*}

$x^2$ is obviously a square number, so each prime factor must appear at least two times (precisely an even number of times). To compensate, $k$ must contain the factors 7, 23 and 487, so $k = 7 \cdot 23 \cdot 487 k'$, with $k'$ a square number.

\subsection{Last equation}

\begin{align*}
    y (3y - 1)  &= 2b \\
    3y^2 - y    &= 2 \cdot 159 \cdot 487 k \\
    3y^2 - y    &= 2 \cdot 7 \cdot 23 \cdot 159 \cdot 487^2 k'
\end{align*}

We have an equation of degree two like this one:
\begin{align*}
    & Ay^2 + By + C = 0 \\
    & A = 3 \\
    & B = -1 \\
    & C = -2 \cdot 7 \cdot 23 \cdot 159 \cdot 487^2 k'
\end{align*}
So we can compute the discriminant

\begin{align*}
    \Delta  &= B^2 - 4 A C \\
            &= (-1)^2 - 4 \cdot 3 \cdot (-2) \cdot 7 \cdot 23 \cdot 159 \cdot 487^2 k' \\
            &= 1 + (2)^3 \cdot 3 \cdot 7 \cdot 23 \cdot 159 \cdot 487^2 k'\\
\end{align*}

We know that there exists a solution (because this challenge can be solved), so $\Delta$ must be positive, and must be a square number. Recall that $k'$ is also a square number.

This equation is of form
\begin{align*}
    X^2 - D \cdot Y^2 = 1
\end{align*}
with $X = \sqrt{\Delta}$ and $Y = \sqrt{k'}$
so it's a Pell-Fermat equation. We can use sympy to solve the Pell-Fermat equation and get the flag:

\inputminted{python}{solve_problemeuh.py}

\end{document}